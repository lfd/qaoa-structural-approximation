\documentclass[aps,rpx,reprint,amsmath,amssymb]{revtex4-2}

\usepackage[active,tightpage,psfixbb]{preview}

\renewcommand{\PreviewBbAdjust}{0pt -4pt 0pt 0pt 0pt}

\usepackage{physics2}
\usephysicsmodule{ab, ab.braket, diagmat}
\usepackage[T1]{fontenc}
\usepackage[utf8]{inputenc}
%\usepackage[libertine]{newtxmath}
\usepackage{tikz}
\usepackage{yquant}
\usepackage{pdfrender}

\usetikzlibrary{calc, fit, shadows.blur}

\newcommand*{\fatten}[1][.4pt]{%
  \textpdfrender{
    TextRenderingMode=FillStroke,
    LineWidth={\dimexpr(#1)\relax},
  }%
}

\newcommand{\vcont}{\raisebox{0.5em}{\(\vdots\)}}
\newcommand{\mangle}{\beta}
\newcommand{\pangle}{\gamma}

\definecolor{lfd1}{HTML}{FFFFFF} % For background use, white is colour 1
\definecolor{lfd2}{HTML}{E69F00}
\definecolor{lfd3}{HTML}{999999}
\definecolor{lfd4}{HTML}{009371}
\definecolor{lfdblue}{HTML}{1f78b4}

\tikzset{group/.style={rectangle, sharp corners,
                       draw=none, fill=#1, nearly transparent}}
\tikzset{compute/.style={fill=lfd2!40}}
\newcommand{\WIDTH}{2.55cm}
\newcommand{\HEIGHT}{1.65cm}
\tikzset{box/.style={rectangle, draw, fill=#1,
                     align=center, blur shadow},
         box/.default=white}
\tikzset{ibox/.style={box, minimum width=\WIDTH,
                     minimum height=\HEIGHT},
         box/.default=white}
\tikzset{varStyle/.style={draw, thin, solid, double}}
\tikzset{dataStyle/.style={draw, semithick, dashed}}
\tikzset{label/.style={fill=black!50, inner sep=0pt, text=white}}
\newcommand{\onbg}[1]{\begin{pgfonlayer}{background}%
                      #1\end{pgfonlayer}}
\def\dist{1.75em}                      
\newcommand{\iterationnf}{\(\textcolor{lfd4}{\circlearrowright}\)}
\newcommand{\iteration}{\fatten[0.7pt]{\iterationnf}}

\begin{document}
\begin{preview}
\begin{tikzpicture}
\pgfdeclarelayer{background}\pgfsetlayers{background,main}

\begin{yquant}[operator/separation=5mm]
  [name=tlqbit] qubit {\(\ket|+>\)} q[2];
  qubit {\vcont\hspace{0.75em}} sep;
  [name=qbit] qubit {\(\ket|+>\)} p[1];

  [style={draw=black},name=mixer] box {$e^{-i \mangle \hat{X}} $} (q, sep, p) ;
  [style={draw=black},name=problem] box {$e^{-i \pangle \hat{C}}$} (q, sep, p) ;

  [name=meas1] measure q;
  [name=meas4] measure sep;
  [name=meas4] measure p;

  [style={draw=black},name=pp] box { \rotatebox{90}{ Sample }} (q, sep, p);
  discard q; discard sep; discard p;
\end{yquant}
    \coordinate (bl) at ([yshift=\dist]mixer.north west);
    \coordinate (br) at ([yshift=\dist]problem.north east);

    \newcommand{\optnode}{\node[draw=none,anchor=south] 
                (optimisation) at ($(bl)!0.5!(br)$) { Optimisation };}
    \optnode

    \coordinate (tr) at ($(problem.north east)!(optimisation.north east)!(br)$);
    \coordinate (tl) at ($(mixer.north west)!(optimisation.north west)!(bl)$);    

    % We redraw the optimisation node to avoid colour mixup
    % with the background
    \node[box, inner sep=0pt, compute, fit=(bl) (tr)] {};
    \optnode
% random instances ->
% sample (partial) target space
% -> optimisation to obtain \mangle, \pangle
% => Run QAOA circuit + sample

    \draw[dataStyle,-Stealth] ($(bl)!(mixer.north)!(br)$) -- (mixer.north);
    \draw[dataStyle,-Stealth] ($(bl)!(problem.north)!(br)$) -- (problem.north);

    \node[box, anchor=south east, align=center] (ts) at 
          ([xshift=-\dist]qbit.south west) { Problem\\Target Space };
    \newcommand{\samplenode}{\node[anchor=south, align=center] (sts) at 
                             ([yshift=\dist]ts.north) { Sample\\\(\tilde{E}(F_{1})\) };}
    \samplenode

    \coordinate (sample_bl) at ($(sts.south west)!(ts.north west)!(sts.south east)$);
    \coordinate (sample_tr) at ($(sts.north west)!(ts.north east)!(sts.north east)$);
    \coordinate (sample_tl) at ($(sts.north east)!(ts.north west)!(sts.north west)$);
    
    % We redraw the sample node to avoid colour
    % mixture with the background
    \node[box, inner sep=0pt, compute, fit=(sample_bl) (sample_tr)] {};
    \samplenode
    \node[anchor=south west,inner sep=2pt] at (sample_bl) 
         { \iteration };
    \draw[varStyle] (ts.north) -- (sts.south);
    
%    \node[draw=none,anchor=south] (landscape) at ($(bl)!0.5!(br)$) { Problem Landscape };
%    \coordinate (tr) at ($(landscape.north east)!(landscape.north east)!(br)$);
%    \draw[black] (bl) -| (tr) (tr) -| (bl);
    
    \node[anchor=south east,inner sep=1pt] at (pp.south east) 
         { \iteration };

    \node[box,anchor=north] (instance) at 
                             ([yshift=-\dist]problem.south) { Instance};
    \draw[dataStyle,-Stealth] (instance.north) -- (problem.south);

    % Non-visible nodes to determine outline 
    % for global portion of the computation
    \node[draw=none, fill=none,
          fit=(ts.south west) (sample_tr)] (global) {};
    \node[draw=none, fill=none, fit=(bl) (tr)] (opt_outline) {};

    \begin{pgfonlayer}{background}
        \fill[group=lfdblue, rounded corners] (global.south west) |- 
                 (opt_outline.north east) |- 
                 ($(opt_outline.south east)!(global.north east)!(opt_outline.south west)$) |- 
                 (global.south west) -- cycle; 
        \node[fill=gray!25, draw=none, rounded corners, fit=(tlqbit-0.north west) (pp.south east) (meas4.south)] (outline) {};
    \end{pgfonlayer}
    
    \draw[dataStyle,-Stealth] (sts.north) |- ($(bl)!0.5!(tl)$);

    \coordinate (center) at ($(global.west |- optimisation.north)!0.5!(outline.east |- optimisation.north)$);
    \node at ([yshift=\dist]center) {\emph{Non-Iterative QAOA}};
\end{tikzpicture}
\end{preview}

\end{document}
