\documentclass[aps,rpx,reprint,amsmath,amssymb]{revtex4-2}

\usepackage[active,tightpage,psfixbb]{preview}

\renewcommand{\PreviewBbAdjust}{0pt 0pt 4pt 0pt}

\usepackage{physics2}
\usephysicsmodule{ab, ab.braket, diagmat}
\usepackage[T1]{fontenc}
\usepackage[utf8]{inputenc}
%\usepackage[libertine]{newtxmath}
\usepackage{tikz}
\usepackage{yquant}
\usepackage{pdfrender}

\usetikzlibrary{calc, fit, shadows.blur}

\newcommand*{\fatten}[1][.4pt]{%
  \textpdfrender{
    TextRenderingMode=FillStroke,
    LineWidth={\dimexpr(#1)\relax},
  }%
}

\DeclareMathOperator*{\Cov}{Cov}
\DeclareMathOperator*{\Var}{Var}

\newcommand{\vcont}{\raisebox{0.5em}{\(\vdots\)}}
\newcommand{\mangle}{\beta}
\newcommand{\pangle}{\gamma}

\definecolor{lfd1}{HTML}{FFFFFF} % For background use, white is colour 1
\definecolor{lfd2}{HTML}{E69F00}
\definecolor{lfd3}{HTML}{999999}
\definecolor{lfd4}{HTML}{009371}
\definecolor{lfdblue}{HTML}{1f78b4}

\tikzset{group/.style={rectangle, sharp corners,
                       draw=none, fill=#1, nearly transparent}}
\tikzset{compute/.style={fill=lfd2!40}}
\newcommand{\WIDTH}{2.55cm}
\newcommand{\HEIGHT}{1.65cm}
\tikzset{box/.style={rectangle, draw, fill=#1,
                     align=center, blur shadow},
         box/.default=white}
\tikzset{ibox/.style={box, minimum width=\WIDTH,
                     minimum height=\HEIGHT},
         box/.default=white}
\tikzset{varStyle/.style={draw, thin, solid, double}}
\tikzset{dataStyle/.style={draw, semithick, dashed}}
\tikzset{label/.style={fill=black!50, inner sep=0pt, text=white}}
\newcommand{\onbg}[1]{\begin{pgfonlayer}{background}%
                      #1\end{pgfonlayer}}
\def\dist{1.75em}                      
\newcommand{\iterationnf}{\(\textcolor{lfd4}{\circlearrowright}\)}
\newcommand{\iteration}{\fatten[0.7pt]{\iterationnf}}

\pgfmathsetseed{19101978}

\newcommand{\blurellipse}[2]{%
%\draw[draw=none, blur shadow] (0,0) ellipse (#1 and #2);
\node[rectangle, anchor=south west, draw=none, blur shadow, rounded corners, minimum width=2*#1, minimum height=2*#2] at (-#1,-#2) {};
}

\newcommand{\randellipse}[3]{%
%\draw (0,0) ellipse (#1 and #2);
\node[rectangle, anchor=south west, draw, rounded corners, minimum width=2*#1, minimum height=2*#2] at (-#1,-#2) {};
\clip (0,0) ellipse (#1 and #2);
\foreach \p in {1,...,#3} {
    \fill[black] (#1*rand,#2*rand) circle (0.35pt);
}}

\newcommand{\invisellipse}[3]{%
\draw[draw=none,fill=none] (0,0) ellipse (#1 and #2);
}

\newcommand{\sublabel}[2]{\node[anchor=south west] at (#1.south west) {\(\bm{#2}\)};}
\newcommand{\flabel}[2]{\sublabel{qaoa#1}{F^{(#2)}_{1}(\mangle, \pangle)}}

\begin{document}
\begin{preview}
\begin{tikzpicture}
    \pgfdeclarelayer{background}\pgfsetlayers{background,main}
    \node[matrix, ibox, anchor=north west] (instance1) at (0,0) {
        \randellipse{1.5em}{0.75em}{20} &&
        \randellipse{1em}{1em}{50} \\
        & \randellipse{0.75em}{1em}{120} &\\
    };

    \node[matrix, ibox, anchor=north west] (instance2) at (3,0) {
        \randellipse{1.5em}{0.75em}{120} &&
        \randellipse{1em}{1em}{50} \\
        & \randellipse{0.75em}{1em}{20} &\\
    };

    \node[matrix, ibox, anchor=north west] (instance3) at (6,0) {
        \randellipse{1.5em}{0.75em}{20} &&
        \randellipse{1em}{1em}{120} \\
        & \randellipse{0.75em}{1em}{50} &\\
    };

    \coordinate (bl) at ([yshift=0.5em]instance1.north west);
    \coordinate (tr) at ([yshift=2em]instance3.north east);
    \node[label, fit=(bl) (tr)] (topi) { };
    \node[anchor=center, text=white] at (topi.center) { Instance Hamming Structures };
    
    \node[matrix, ibox, anchor=north west] (global) at (3,-3) {
        \blurellipse{1.5em}{0.75em} &&
        \blurellipse{1em}{1em} \\
        & \blurellipse{0.75em}{1em} &\\
    };

    \node[anchor=south west] at (global.south west) {
      \(\bm{w_{d_1,d_2}(\pangle)}\)
    };
    \node[anchor=south, fill=black!50, text=white, align=center, rotate=90] (gps) at ([xshift=-0.5em]global.west) { Global\\Problem\\Structure };

    \coordinate (center) at ($(instance2.south)!0.5!(global.north)$);
    \draw[draw=none] (instance1.south |- center) -- (center)
                     node[below,pos=0.7] {\small\emph{Sampling}};
    \draw[draw=none] (instance3.south |- center) -- (center)
                     node[below,pos=0.725] (analytic) {\small\emph{Analytic}};
    
    \draw[-Stealth, dataStyle] (instance2.south) -- (global.north);
    \draw[-Stealth, dataStyle] (instance3.south) |- (center) -- (global.north);
    \draw[-Stealth, dataStyle] (instance1.south) |- (center) -- (global.north);

    \newcommand{\lscapeimg}[2]{\node[opacity=0.5] at (#1.center) { \includegraphics[width=\WIDTH]{figures/generated/landscape#2.pdf} };}
    \node[ibox, anchor=north west] (avginf) at (3,-6) {};
    \lscapeimg{avginf}{4}
    \sublabel{avginf}{\textcolor{lfd2}{\tilde{E}(F_{1}(\mangle, \pangle)}}
    

    \draw[-Stealth] (global.south) -- (avginf.north);

    %%%%%%%%% Combinatorial Decision Problems %%%%%%%%%%%%%%%%%
    \newcommand{\binimg}[2]{\node[opacity=0.5] at (#1.center) { \includegraphics[width=\WIDTH]{figures/generated/binary#2.pdf} };}
    \node[ibox, anchor=north west] (landscape1) at ($(-3,0) - (1em,0)$) { };
    \binimg{landscape1}{1}
    \node[anchor=south west] at (landscape1.south west) {\(\bm{c^{(3)}(\vec{z})}\)};

    \node[ibox, anchor=north west] (landscape2) at ($(-6,0) - (1em,0)$) { };
    \binimg{landscape2}{2}
    \node[anchor=south west] at (landscape2.south west) {\(\bm{c^{(2)}(\vec{z})}\)};

    \node[ibox, anchor=north west] (landscape3) at ($(-9,0) - (1em,0)$) { };    
    \binimg{landscape3}{3}
    \node[anchor=south west] at (landscape3.south west) {\(\bm{c^{(1)}(\vec{z})}\)};

    %%%%%%%%%%%% QAOA Optimisation Landscapes %%%%%%%%%%%%%%%%%%
    \node[ibox, anchor=north west] (qaoa1) at ($(-3,-3) - (1em,0)$) { };
    \lscapeimg{qaoa1}{1}
    \flabel{1}{3}

    \node[ibox, anchor=north west] (qaoa2) at ($(-6,-3) - (1em,0)$) { };
    \lscapeimg{qaoa2}{2}
    \flabel{2}{2}

    \node[ibox, anchor=north west] (qaoa3) at ($(-9,-3) - (1em,0)$) { };    
    \lscapeimg{qaoa3}{3}
    \flabel{3}{1}

    \foreach\i in{1,2,3} { \draw[-Stealth] (landscape\i.south) -- (qaoa\i.north); }

    \coordinate (bl) at ([yshift=0.5em]landscape1.north east);
    \coordinate (tr) at ([yshift=2em]landscape3.north west);
    \node[label, fit=(bl) (tr)] (topl) { };
    \node[anchor=center, text=white] at (topl.center) { Instance Landscapes };  

    \node[ibox, anchor=north west] (avglscape) at ($(-6,-6) - (1em,0)$) { };
    \lscapeimg{avglscape}{4}
    \sublabel{avglscape}{\textcolor{lfd4}{E(F_{1}(\mangle, \pangle)}}

    \coordinate (center) at ($(qaoa2.south)!0.5!(avglscape.north)$);
    \draw[-Stealth] (qaoa1.south) |- (center) -- (avglscape.north);
    \draw[-Stealth] (qaoa2.south) -- (avglscape.north);
    \draw[-Stealth] (qaoa3.south) |- (center) -- (avglscape.north);

    \newcommand{\avglabel} { \node[anchor=north, fill=black!50, text=white, align=center] (avglabel) at ([yshift=-0.5em]avglscape.south) { Expected\\Landscape };}
    \avglabel
    \coordinate (tl) at ($(avglscape.north west)!(avglabel.north west)!(avglscape.south west)$);
    \coordinate (br) at ($(avglscape.north east)!(avglabel.south west)!(avglscape.south east)$);
    \node[fill=black!50, draw=none, inner sep=0pt, fit=(tl) (br), align=center] { };
    \avglabel

    \newcommand{\explabel} { \node[anchor=north, fill=black!50, text=white, align=center] (explabel) at ([yshift=-0.5em]avginf.south) { Approximated\\Landscape };}
    \explabel
    \coordinate (tl) at ($(avginf.north west)!(avglabel.north west)!(avginf.south west)$);
    \coordinate (br) at ($(avginf.north east)!(avglabel.south west)!(avginf.south east)$);
    \node[fill=black!50, draw=none, inner sep=0pt, fit=(tl) (br), align=center] { };
    \explabel

%    \coordinate (center) at ($(landscape2.south)!0.5!(avglscape.north)$);
%    \draw[-Stealth] (landscape1.south) |- (center) -- (avglscape.north);
%    \draw[-Stealth] (landscape2.south) -- (avglscape.north);
%    \draw[-Stealth] (landscape3.south) |- (center) -- (avglscape.north);

    \draw[Stealth-Stealth, varStyle] (avglscape.east) -- (avginf.west) node[midway, below, font=\footnotesize] { \(\ab|\textcolor{lfd4}{E\ab(F_1)} - \textcolor{lfd2}{\tilde{E}\ab(F_1)}| \leq \sqrt{\Var\ab(\ab|T|) \Var\ab(\overline{\ab|c_k|^2})}\)} node[midway, above] {Bounded Difference};

    \coordinate (linetop) at ($(topi.north east)!0.5!(topl.north west)$);
    \draw[black,thick,dotted]  (linetop) -- ($(global.south east)!(linetop)!(global.south west)$);

    \coordinate (bl) at ($(avglscape.north west)!(avglabel.south west)!(avglscape.south west)$);

    \coordinate (corner) at ($(avglscape.north west -| gps.north) + (-5pt, 5pt)$);
    \coordinate (blshift) at ([xshift=-5pt, yshift=-5pt]bl);
    \begin{pgfonlayer}{background}
        \fill[group=lfdblue, inner sep=2pt, rounded corners] 
              (blshift) |- (corner) |- ([yshift=5pt]analytic.north east) |- 
              (blshift) -- cycle;
    \end{pgfonlayer}
\end{tikzpicture}
\end{preview}

\end{document}
